$<$$<$$<$$<$$<$$<$$<$ H\+E\+A\+D \section*{F\+S\+M\+Driver Controller }

\subsection*{Description }

A driver implementation using \hyperlink{md_FSM_Description}{Finite State Machine}(F\+S\+M) to compete at Simulation Car Racing Championship at T\+O\+R\+C\+S Simulator environment, \href{http://arxiv.org/abs/1304.1672}{\tt Simulated Car Racing Championship}. This file describes the basic configuration to compile and run the the F\+S\+M\+Driver controler, including libraries and commands to linux environment(\+Ubuntu).

\subsection*{Copyright }

This work use the src code developed by Daniele Loiacano.\+To understand the basic functions used at F\+S\+M\+Driver(such as drive, enter, shutdown) take a look at the folder src.

\subsection*{Requirement }

A makefile is used to minimize the commands needed to compile and run the program. Make sure to install the following programs and libraries\+:


\begin{DoxyItemize}
\item Torcs Simulator(at least 1.\+3.\+6). More information at\+: \href{http://torcs.sourceforge.net/;}{\tt http\+://torcs.\+sourceforge.\+net/;}
\item Build-\/\+Essential\+: it contein dpkg, g++(at least 4.\+6 to provide support to c++11 standart), libc-\/dev and make;
\item Cmake\+: it is important to build the program, reducing the effort to compile it;
\item Doxygen(optional)\+: a documentation programm to organize the comments and code properly;
\end{DoxyItemize}

The packages can be installed following the command\+: 
\begin{DoxyCode}
1 $ sudo apt-get install cmake build-essential
\end{DoxyCode}
 To install doxygen\+: 
\begin{DoxyCode}
1 $ git clone https://github.com/doxygen/doxygen.git (optional)
2 $ cd doxygen
3 $ ./configure
4 $ make
5 $ make install
\end{DoxyCode}


\subsection*{Usage }

The F\+S\+M\+Driver can drive at T\+O\+R\+C\+S Simulator. It is build using makefile.

\subsubsection*{Makefile}

Assuming that the previous requirements are fulfiled, the driver code can be build following the steps bellow\+:


\begin{DoxyCode}
1 $ cd ~/fsmdriver/build
2 $ cmake ..
3 $ make
\end{DoxyCode}


The compiling process creates an executable named F\+S\+M\+Driver, to run it and torcs type the commands\+: 
\begin{DoxyCode}
1 $ cd ../bin
2 $ ./FSMDriver & torcs
\end{DoxyCode}


At the Torcs\textquotesingle{} graphic interface, click\+: race -\/$>$ Quick Race -\/$>$ New Race.

\subsection*{Documentation }

To better understand this code it is possible to use doxygen to generate an html file containing the this documentation. Assuming that doxygen is installes follow the steps bellow\+:


\begin{DoxyCode}
1 $ cd ~/fsmdriver
2 $ doxygen -g D
3 $ doxygen D
4 $ doxygen doxyfile
5 $ firefox html/index.html
\end{DoxyCode}




 \section*{new\+F\+S\+M Controller }

This is an implementation of a \href{http://arxiv.org/abs/1304.1672}{\tt S\+C\+R} client controller using a Finite State Machine. Code was tested on Ubuntu 14.\+04 with T\+O\+R\+C\+S 1.\+3.\+6.

\subsection*{Dependencies }


\begin{DoxyItemize}
\item \href{http://torcs.sourceforge.net/}{\tt T\+O\+R\+C\+S} 1.\+3.\+4 (or higher). Be sure to install it from the \href{http://torcs.sourceforge.net/index.php?name=Sections&op=viewarticle&artid=3#linux-src-all}{\tt Source Package}, and not a binary distribution (better yet, use the instructions below for installing the S\+C\+R patch).
\item \href{http://sourceforge.net/projects/cig/files/SCR%20Championship/Server%20Linux/}{\tt S\+C\+R patch} as described in \char`\"{}\+Simulated Car Racing Championship Competition Software Manual\char`\"{} (\href{http://arxiv.org/pdf/1304.1672v2}{\tt v2 -\/ April 2013}),
\end{DoxyItemize}

\subsection*{Usage }

A \href{http://www.cmake.org/documentation/}{\tt C\+Makefile} is provided in hopes of simplifying the build process (implying in additional {\ttfamily make} and {\ttfamily cmake} dependencies). Assuming you\textquotesingle{}ve downloaded the code, F\+S\+M\+Driver can be compiled with\+:


\begin{DoxyCode}
1 $ cd fsmdriver/build
2 $ cmake ..
3 $ make
\end{DoxyCode}


This creates the executable {\ttfamily F\+S\+M\+Driver}. To test it (assuming you started a properly configured race), just run\+:


\begin{DoxyCode}
1 ./bin/FSMDriver
\end{DoxyCode}


\subsection*{Documentation }

Documentation (and further explanations) for the code is available in the source code files, and can be extracted via \href{www.doxygen.org}{\tt Doxygen}. \begin{quote}
\begin{quote}
\begin{quote}
\begin{quote}
\begin{quote}
\begin{quote}
\begin{quote}
new\+Fsm\end{quote}
\end{quote}
\end{quote}
\end{quote}
\end{quote}
\end{quote}
\end{quote}
