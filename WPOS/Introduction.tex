\section{Introduction}
\IEEEPARstart{D}{igital} games provide a great test bed for experimentation and 
study of Artificial Intelligence (AI), and there has been a growing interest in 
applying AI in them, regardless of genre \cite{simon2008}. Some applications 
include controlling non-playable characters \cite{stanley2005}, path planning \cite{freitas2012}, human pose recognition \cite{shotton2011}, and others. 

Electronic games also present a well defined environment, which may simulate 
extremely complex situations such as flying an airplane or controlling a car 
\cite{scr2009}. One such simulator is TORCS, \emph{The Open Racing Car Simulator}
\cite{TORCS} whose input and output are constrained by the software's characteristics, providing an ideal benchmark for comparing AI solutions for creating car controllers.

 This paper is organized as follows: Section 2 presents the details on TORCS and 
 the Simulated Car Racing Championship and some background on finite state machines,
 Section 3 describes the implementation of the FSMDriver, Section 4 presents 
 initial experimental results, and Section 5 provides concluding remarks.
